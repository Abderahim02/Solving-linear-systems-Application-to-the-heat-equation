\documentclass{article}
\usepackage[french]{babel}
\usepackage[utf8]{inputenc}
\usepackage{graphicx}
\usepackage{amsmath}
\usepackage{subcaption}
\usepackage[T1]{fontenc}

%%%%%%%%%%%%%%%% Lengths %%%%%%%%%%%%%%%%
\setlength{\textwidth}{16cm}
\setlength{\evensidemargin}{0.3cm}
\setlength{\oddsidemargin}{0.3cm}

%%%%%%%%%%%%%%%% Variables %%%%%%%%%%%%%%%%
\def\projet{2}
\def\titre{Résolution de systèmes linéaires, application à l'équation de la chaleur}
\def\groupe{1}
\def\equipe{5}
\def\responsible{sdribialaou}
\def\secretary{cdomas}
\def\others{mbouhaja, alagraoui001}

\begin{document}

%%%%%%%%%%%%%%%% Header %%%%%%%%%%%%%%%%
\noindent\begin{minipage}{0.98\textwidth}
  \vskip 0mm
  \noindent
  { \begin{tabular}{p{7.5cm}}
      {\bfseries \sffamily
        Projet \projet} \\ 
      {\itshape \titre}
    \end{tabular}}
  \hfill 
  \fbox{\begin{tabular}{l}
      {~\hfill \bfseries \sffamily Groupe \groupe\ - Équipe \equipe
        \hfill~} \\[2mm] 
      Responsable : \responsible \\
      Secrétaire : \secretary \\
      Codeurs : \others
    \end{tabular}}
  \vskip 4mm ~

  ~~~\parbox{0.95\textwidth}{\small \textit{Résumé~:} \sffamily Le but de ce projet consiste à implémenter des algorithmes de résolution de systèmes linéaires de grande taille, et à les appliquer à un problème de résolution d’équation aux dérivées partielles. Dans ce projet, on considère uniquement des systèmes linéaires symétriques, définis positifs et creux (ne comportant que relativement peu d’éléments non nuls), et on exploite ces trois propriétés pour obtenir des algorithmes plus efficaces. }
  \vskip 1mm ~
\end{minipage}

%%%%%%%%%%%%%%%% Main part %%%%%%%%%%%%%%%%

\section{Résolution de systèmes linéaires}
% Soit un système linéaire $A.x = b$ où $A$ est une matrice carrée de taille $(n,n)$ et $b$ et $x$ deux vecteurs colonnes de taille $n$ avec $n$ un entier naturel. La résolution de ce système consiste à trouver des valeurs du vecteur $x$ qui le vérifie. Pour cela, les études en algèbre linéaire fournissent plusieurs méthodes qui peuvent être directes ou itératives.

% Les méthodes directes sont souvent plus précises mais peuvent être plus coûteuses en termes de temps de calcul et de mémoire, tandis que les méthodes itératives sont plus efficaces pour les systèmes linéaires très grands mais nécessitent souvent une bonne approximation initiale et une méthode de convergence adaptée.

% Dans ce projet, on s'intéresse aux deux méthodes avec l'exemple de la décomposition de cholesky pour les méthodes directes et la méthode du gradient conjugué pour les méthodes itératives.
Pour résoudre un système linéaire $A.x = b$, ils existent deux approches principales: les méthodes directes et les méthodes itératives. Les méthodes directes sont plus précises mais peuvent être plus coûteuses en termes de temps de calcul et de mémoire, tandis que les méthodes itératives sont plus efficaces pour les systèmes linéaires très grands, mais nécessitent souvent une bonne approximation initiale et une méthode de convergence adaptée. Nous explorons les deux approches dans ce projet, avec la décomposition de $Cholesky$ comme exemple de méthode directe et la méthode du gradient conjugué comme exemple de méthode itérative.
\subsection{Décomposition de Cholesky}
Soit $(E) : A.x = b$ un système linéaire où $A$ est une matrice symétrique définie positive de taille $(n,n)$
et $b$ et $x$ deux vecteurs colonnes de taille $n$ avec $n$ un entier naturel.
L'algorithme de $cholesky$ consiste à factoriser la matrice $A$ sous la forme d’un produit $ T^{t}.T$ où $T$
est une matrice triangulaire inférieure obtenue par les formules suivantes :

\[ \large t_{i, j} = \left\{
    \begin{array}{cr}
        \rule{0pt}{2.5ex} \sqrt{ a_{i,i} - \sum_{k=1}^{i-1} t_{i,k}^{2}} & \mbox{si } j=i \\
        \rule{0pt}{4ex} \frac{ \LARGE a_{i,j} - \sum_{k=1}^{i-1} t_{i,k}t_{j,k} } {a_{i,i}}& \mbox{si i $\leq$ j}
    \end{array}
\right.\]

\subsection{Méthode du gradient conjugué}


Afin d'obtenir la solution d'un système d'équation linéaires $A.x = b$ (où A est 
une matrice symétrique définie positive et b un vecteur), on utilise la méthode du gradient 
conjugué.


Cette méthode consiste à trouver une solution $x$ en minimisant l'erreur relative 
$rsnew$ ($rsnew~=~((b-A.x)^{T}(b-A.x))^{\frac{1}{2}}$) d'une manière itérative, 
elle calcule au début une direction de descente en choisissant  ($p=b-Ax$) comme 
première direction de recherche, puis il est mis à jour à chaque étape on calcule une nouvelle 
direction en combinant la direction précédente avec la nouvelle direction de 
recherche ($r=r+\beta*p$). Après un nombre fini d'itérations (la taille du vecteur b)
notre algorithme converge vers la solution optimal avec une erreur relative minimal.


\paragraph{Exemple}
On montre dans cette partie la valeur du vecteur x à chaque itérations dans la fonction
du gradient conjugué. Donc on prend une matrice d'ordre 3 et un vecteur b qui sont définie comme suit:

$A=$\begin{pmatrix}
  3&2&-5 \\ 1&5&10 \\ 6&4&5
\end{pmatrix}
$b=$\begin{pmatrix}
  20 
  \\ -7 \\
    5
\end{pmatrix}


\section{Application à l'équation de la chaleur}

\end{document}
